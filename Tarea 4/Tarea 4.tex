\documentclass[letterpaper,12pt]{article}
\usepackage[utf8]{inputenc}
\usepackage{amsmath}
\usepackage{amssymb}
\usepackage{mathrsfs}
\usepackage[dvipsnames]{xcolor}
\usepackage{wasysym}
\usepackage{textcomp}
\usepackage{fancyhdr}
\usepackage[left=2cm,right=2cm,top=2.5cm,bottom=2cm]{geometry}
%--------------------------------------------------------------
\pagestyle{fancy}
\begin{document}
\hspace{3.5cm}{\textbf{\huge\textsf{\textcolor{cyan}{La} 
\textcolor{red}{receta} \textcolor{RubineRed}{de la} \textcolor{YellowOrange}{perdición}}}}
\rfoot{Rodriguez Vega Nayeli}
%------------------------------------------------------------------------------------------------
\section{Fisíca}
\begin{itemize}
    \item[$\heartsuit$] Esta fórmula nos ayuda a calcular la {\textbf{aceleración angular}}$ \left(\frac{rad}{s^{2}}\right)$, no obstante, para utilizarla debemos de conocer la velocidad angular inicial ($w_f$), la velocidad angular final ($w_i$) y el tiempo ($t$).
    \textcolor{magenta}{$$\alpha= \frac{\triangle w}{t}= \frac{w_f-w_i}{t}$$}

    \item[$\clubsuit$]{\textbf {La ecuación de Bernoulli:}} Conociendo la densidad del líquido ($\rho$), la velocidad que fluye el líquido($v$), la aceleración de la gravedad $\left(g=9.81\frac{m}{s^{2}}\right
    )$, la presión ($p_1,p_2$) y la altura de la vertical relativa del sistema, nos ayudaria a conocer los trabajos mecánicos vistos en el sistema hidráulico, que se mueve únicamente con la gravedad dependiendo del fluido que se meta.
    \textcolor{blue}{$$p_1+\frac{1}{2}\rho v^{2}_1+pgh_1=p_2+\frac{1}{2}\rho v^{2}_2+\rho gh_2$$}
    
    \item[$\sun$] Para saber la {\textbf{fuerza de electrostática}} de dos cargas puntuales, debemos de conocer la distancia que las separa ($r$) el valor de las cargas ($  q_1,q_2$) y la constante electrostática$\left (k=\frac{1}{4 \pi \epsilon_0}\right)$.
    \textcolor{magenta}{$$\vec{F}=k\frac{q_1q_2}{r^{2}}$$}
    
    \item[$\smiley$]Para conocer la {\textbf{capacitancia}} equivalente que tiene un circuito en serie debemos de conocer el valor de las capacitancias ($C_1, C_2,...C_n$).
    
    \textcolor{blue}{$$C_{eq}=\frac{1}{\frac{1}{C_1}+\frac{1}{C_2}+...+\frac{1}{C_n}}$$}
    
    \item[\textmusicalnote]Esta fórmula nos ayuda a encontrar el {\textbf{momento de inercia}} en un cilindro hueco, no obstante para poderla ocupar necesitamos el valor del radio del hueco($R_1$), el radio del cilindro general($R_2$) y la masa del cilindro($M$).
    
    \textcolor{magenta}{$$I=\frac{1}{2}M(R^{2}_1+R^{2}_2)$$}
    
     \item[$\heartsuit$] {\textbf{El teorema del virial:}} Este teorema nos muestra una relación estadística entre los valores medios de varias magnitudes mecanicas con respeto al tiempo. La ecuación general que nos da el teorema, relaciona el promedio a lo largo del tiempo de la energía cinética total de un sistema estable de partículas discretas, unidas por fuerzas potenciales, conla energía potencial total del sistema.
    \textcolor{blue}{$$\langle T\rangle=-\frac{1}{2}\sum_{n=0}^{N}\langle {\textbf F}_k \cdot{\textbf r}_k\rangle $$}
    En donde: \\
    $\langle T\rangle$=Energía cinética.
    
    ${\textbf F}_k$=Fuerza aplicada en la particula.
     
      ${\textbf r}_k$=Posición.
      
    \item[$\clubsuit$]{\textbf {La ley universal grabitacion:}} En esta fórmula nos empieza a describir la interacción gravitatoria entre los cuerpos, es decir establece que si una partícula que tiene cierta masa ($m_1$) está ejerciendo una fuerza de atracción sobre otra partícula de masa ($m_2$) que es directamente proporcional al producto de las dos masas e inversamente proporcional al cudrado de la distancia ($r^{2}$) que las separa.
    
    \textcolor{magenta}{$$\vec{F_g}=G\frac{m_1m_2}{r^{2}}\vec{e_r}$$}
%-------------------------------------------------------------
     \item[$\sun$]{\textbf {La segunda ley de Newton:}} Esta ley establece que las aceleraciones que experimenta un cuerpo son proporcionales a las fuerzas que recibe, la cual se llega a expresar como:
     $$\vec{F_N}=\frac{d\vec {p}}{dt}=\frac{d(m\vec {v})}{dt}=m\frac{d\vec {v}}{dt}+\vec {v}\frac{dm}{dt}={m\vec {a}}$$
     {\textcolor{blue}{$$\therefore \vec{F_N}=m\vec {a}$$}}
     En donde:
     
     $\vec{F_N=}$ Es la fuerza neta.
     
     $\vec{p}=$ Momento.
     
     $\vec{v}=$ Velocidad.
     
     $m=$ Masa.
     
     $t=$ Tiempo.
     
 %----------------------------------------------------------
     \item[$\smiley$]{\textbf {La primera ley de kepler:}} Primeramente estra ley nos dice que todos los planetas se mueven en órbitas elípticas con el Sol en uno de los focos, por lo tanto esta ley la llegan a llamar la ley de la orbita, por lo cual se llega a expresar de la siguiente manera:
     
      \textcolor{magenta}{$$r=\frac{p}{1+\varepsilon cos(\phi_0-\theta_0)}$$}
      
      En donde:
      
      $\varepsilon$= Es la excentricidad, pero si $\varepsilon<0 $ la ecuacion es una eclipse, no obstante si $\varepsilon=0$ la orbita es un círculo con centro en el centro de la tierra.
      
      $\theta_0$= Orientación de la elipse con respecto al plano orbital (ejes $x_0$ e $y_0$).
      
      $p=\frac{h^{2}}{\mu}$ en donde $h= $m¿Magnitud angular del vector momentum.
     
     \item[\textmusicalnote]{\textbf{El oscilador armónico:}}     Hace referencia cuando el cuerpo se mueve hacia uno y otro lado respeto a una pocisión de equilibrio, pero esta sometido a una fuerza recuperadora, que tiende a devolverlo al punto de equilibrio estable, con una intensidad proporcional a la separación respecto de dicho punto.
     
      \textcolor{blue}{$${\textbf{$\Psi$}}(t)=\hat{\Psi}e^{i(wt \pm \rho)}\equiv \hat{\Psi} cos (wt \pm \rho)$$}
      
    En donde $\hat{\Psi}$ es la aplitud.
%---------------------------------------------------------------
     \item[$\heartsuit$]{\textbf {Ecuación de la onda:}} Esta ecuación nos ayuda a describir el movimiento de muchas clases de onda, por lo tanto la forma general de la formula es $\Box u=0$ (donde $\Box$ se denomina D'Alambertiano).
     
     \textcolor{magenta}{$$\Box u= \bigtriangledown^{2}u-\frac{1}{v^{2}}\frac{\partial^{2}u}{\partial t^{2}}=\frac{\partial^{2}u}{\partial x^{2}}\frac{\partial^{2}u}{\partial y^{2}}\frac{\partial^{2}u}{\partial z^{2}}-\frac{1}{v^{2}}\frac{\partial^{2}u}{\partial t^{2}}=0$$}
     En donde: 
     
     $u$=Onda
     
     $v$=Velocidad de propagación
     
     \item[$\clubsuit$]{\textbf {Ley de Gauss:}} Esta ley nos permite calcular de una forma simple el modulo del campo electrico, cuando conocemos la distribución de cargas con simetría esféricas o cilindricas.
     Si un vector $\vec{S}$ es un vector que tienen por modulo el area de dicha supeficie, la dirección es perpendicular al plano que la contienen, no obstante cuando el vector campo $\vec{E}$ y el vector superficie $\vec{S}$ son perpendiculares el flujo es cero,por lo tanto si el campo no es constante o la superficie no es plana, se calcula que el flujo a través de cada elemento $\vec{dS}$ de superficie,$\vec{E} \cdot \vec{dS}$, entonces el flujo de la superficie S es:
     
    $$\Phi_E=\int_{s} \vec{E}\cdot \vec{dS}$$
     
     Pero si el  flujo del campo eléctrico pasa través de cualquier superficie cerrada es igual a la carga ($q$ contenida dentro de la superficie, dividida por la constante $\varepsilon _{0}$.
     
     $$\Phi_E=\oint \vec{E}\cdot \vec{dS}=\frac{q}{\varepsilon _{0}}$$
     {\textcolor{blue}{$$\therefore \Phi=\frac{q}{\varepsilon _{0}}$$}}
     En donde:
     
     $\Phi_3$=Flujo eléctrico a travéz de una véz de una superficie cerrada $S$ encerrando cualquier volumen V
     
     $q$=Carga total encerrada,encerrado  detro de V
     
     ${\varepsilon _{0}}$=Constante Eléctrica
     
     \item[$\sun$]{\textbf {Ley de Poiseulle:}}Permite determinar el flujo de un fluido a través de un cilindro de longitud ($L$), apartir del caudal de volumén que esta dado por la diferencia de presión ($P_1-P_2$) por la resistencia viscosa.
     
     $$\Phi_V=\frac{dV}{dt}=v_{media \pi r^{2}}=\frac{\pi r^{4}}{8\eta}\left(-\frac{dP}{dz}\right)=\frac{\pi r^{4}}{8\eta}\frac{\vartriangle P}{L}=\frac{\pi r^{4}  (P_1-P_2)}{8\eta L}$$
     
     {\textcolor{magenta}{$$\therefore \Phi_V=\frac{\pi r^{4}  (P_1-P_2)}{8\eta L}$$}}
     
     \item[$\smiley$]{\textbf {Presión Dinamica:}} Se refiere cuando los fluidos se mueven en un conducto, la inercia del movimiento  produce un incremento adicional de la presión estatica al chocar sobre un área perpendicular, por lo tanto a esa fuerza que se produce por la preción es lo que conocemos como dinámica.Esta formula llegamos a ocupar para calcular la pédida  de presión en la tuberías.
     
     \textcolor{blue}{$${Pd=\frac{\rho V}{2g}}$$}
     
     En donde:
     
    $Pd$=Presión dinámica en pascales.
    
    $V$=Velocidad de flujo.
    
    $\rho$=Densidad del flujo en masa.
    
    $g$=La constante de la grabedad $9.81\frac{m}{s^{2}}$.
    
    \item[\textmusicalnote]{\textbf {Velocidad del sonido en gases:}} Si desamos encontar la velocidad del sonido en un gas ideal debemos de conocer la masa molecular del gas ($M$), la constante adiabática($\gamma$), la constante del gas universal$\left(R=8,314\frac{J}{molk}\right)$ y la temperatura absoluta($4$).
    
     \textcolor{Magenta}{$${V_{sonido}=\sqrt{\frac{ \gamma RT}{M}}}$$}
    
     
  \end{itemize} 
%-------------------------------------------------------------
\section{Geometría analítica}
%-------------------------------------------------------------
\begin{itemize}
\item[$\heartsuit$]{\textbf{Ley del seno:}} En esta ley nos decribe que existe la relación entre los lados y angulos de un tiangulo no rectangulo, en otras palabras que estatablece que la relación de la longitud de un lado de un triangulo al seno del angulo opuesto,ese lado que agarramos es igual para todos los lados y angulos del triangulo dado.
\textcolor{blue}{$$\frac{a}{seno(\alpha)}+\frac{b}{seno(\beta)}+\frac{c}{seno(\gamma)}$$}
%----------------------------------------------------------------
\item[$\clubsuit$]{\textbf{Ley de los cosenos:}} Esta ley hace referencia a la relación entre longitudes de los lados de un triangulo con respeto al coseno de su angulo intermedio,por lo tanto esta ley nos dice que el cuadrado de un lado es igual a la suma de los cuadrados de los otros lados menos el doble del producto de estos lados y el coseno del ángulo intermedio. Esta ley es usada cuando queremos encontrar la longitud de un tercer lado y conocemos las longitudes de los dos lados y el ángulo entre ellos.

\textcolor{magenta}{$$a^{2}=b^{2}+c^{2}-2bc Cos(\alpha)$$}
\textcolor{magenta}{$$b^{2}=c^{2}+a^{2}-2ca Cos(\beta)$$}
\textcolor{magenta}{$$c^{2}=a^{2}+b^{2}-2ab Cos(\gamma)$$}

%----------------------------------------------------------
\item[$\sun$] {\textbf{La distancia entre dos puntos:}} Al tener dos puntos A($x_1,y_1$) y B($x_2,y_2$)en el plano cartesiano la distancia entre los puntos corresponde al valor absoluto de la diferencia de sus abscisas (x) y ordenadas(y).

\textcolor{blue}{$$d_{AB}=\sqrt{ {\lvert x_2-x_1 \rvert}^{2}+{\lvert y_2-y_1 \rvert}^{2}}$$}
%----------------------------------------------------------
\item[$\smiley$] {\textbf{Coordenadas polares a coordenadas cartesianas:}} Sabemos que las coordenadas polares (r,$\theta$) estan conformada por la distancia del punto y por su angulo, por lo tanto para poder transformala a coordenadas cartesianas(x,y) tenemos que descomponer la distancia contemplando el angulo en sus componenetes.
\textcolor{magenta}{$$x=r Cos(\theta)$$}
\textcolor{magenta}{$$y=r Sen(\theta)$$}
%-------------------------------------------------------

\section{Química}
\begin{itemize}
    \item[$\heartsuit$]{\textbf{ 2-acetilpiridina:}} Es un compuesto organico, liquido viscoso incoloro que se usa principalmente como un aromatizante y como saborizante para alimentos como para el sabor de las tortillas o palomitas. Esta sustancia esta compuesta por Carbono (C),Hidrógeno (H), Oxígeno (O) y Nitrógeno (N).
    
    \textcolor{blue}{$$CH_3COC_5H_4N$$}
%---------------------------------------------------------------
\item[$\clubsuit$]{\textbf {Propanamida:}} Es una amida que proviene del ácido propiónico y suele ser utilizada elaborar resinas textiles, fármacos, pesticidas entre otras substancias químicas, su formula quimica esta compuesta por Carbono (C),Hidrógeno (H), Oxígeno (O), y  Nitrógeno (N).

\textcolor{magenta}{$$C_3H_7NO$$}

Su formula desarrollada seria:

\textcolor{magenta}{$$CH_3-CH_2-CH-NH_2$$}
%---------------------------------------------
\item[$\sun$]{\textbf{Paraquat:}} Es un herbicida de contacto no selectivo y se a llegado ocupar para controlar las malesas y los pastos invasores en muchos entornos agricolas y noagricolas, igualmente esta sustancia esta compuesta por Carbono (C),Hidrógeno (H), Oxígeno (O), Cloro (Cl), Azufre (S) y  Nitrógeno (N).

\textcolor{blue}{$$C_{12}H_{14}N_2Cl_2C_{12}H_{14}N_2(CH_3SO_4)_2$$}

\item[$\smiley$]{\textbf{Propanamida:Arseniato de plomo:}}Es un polvo pesado, blanco e inodoro, que se utiliza para matar insectos, malezas y roedores, no obstante se considera una sustancia peligrosa para los humanos.Esta sustancia esta conmpuesta por Carbono (C),Hidrógeno (H), Oxígeno (O), Plomo (Pb) y Arsenico (As).

\textcolor{magenta}{$$Pb(AsO_3)_2Pb_3(AsO_4)_2PbHAsO_4Pb(H_2AsO_4)_2Pb_2As_2O_7$$}

\end{itemize}

\section{Estadística}
\begin{itemize}
    \item[$\heartsuit$]{\textbf{Varianza muestral:}} Se refiere al promedio de las desviaciones de los datos con respecto a la media muestral, en otras palabras esta formula se llega usar para representar la variabilidad de un conjunto de datos respecto de la media aritmetica de los mismos.
    
    {\textcolor{magenta}{$$S^{2}=\frac{\sum\limits_{i=1}^{n}(x_i-\overline{x})^{2}}{ n-1}$$}}
    
    En donde:
    
    $S$= Es la varianza.
    
    $x_i$=Término de conjunto de datos.
    
    $\overline{x}$= Media de la muestra.
    
    $n$=Tamaño de la muestra.
\section{Bíologia}
\begin{itemize}
    \item[$\heartsuit$]{\textbf{Fotosíntesis:}} Esta formula quimica explica la manera en que las plantas toman la energia del sol y la utilizan para convertir el dióxido de carbono y agua en  moleculas necesarias para su crecimiento, no obstante este proceso biologico aparte de liberar oxígeno ayuda a mitigar la contaminación ambiental.

\textcolor{blue}{$$6CO_2+H_2O \rightarrow C_6H_{12}O_6+6O_2$$}
En donde: 

\begin{center}
Dióxido de carbono + Agua $\rightarrow$  Glucosa + Oxígeno    
\end{center}

\end{itemize}

 \end{itemize}
    
\end{itemize}
\end{document}
